\documentclass{article}
\usepackage[UTF8]{ctex}
\usepackage[tc]{titlepic}
\usepackage{titlesec}
\usepackage{cite}
\usepackage{caption}
\usepackage{fancyhdr}
\usepackage{booktabs}
\usepackage{graphicx}
\usepackage{geometry}
\usepackage{amsmath}
\usepackage{cases}
\usepackage{amssymb}
\usepackage[section]{placeins}
\usepackage{subfigure}
\usepackage[final]{pdfpages}
\usepackage{float}
\usepackage{tikz}
\usepackage{listings}
\geometry{a4paper,scale=0.8}
\pagestyle{fancy}

\lhead{第 1 次作业\\\today}

\rhead{Assignment 6\\ {\CTEXoptions[today=old]\today}}
\newcommand{\upcite}[1]{\textsuperscript{\cite{#1}}}

\titleformat*{\section}{\bfseries\Large}
\titleformat*{\subsection}{\bfseries\large}

\title{\bfseries Assignment1}
\author{刘兆琦  \quad  学号 2201110045}

\def \intt{\int_0^1}

\begin{document}
\maketitle
% \setcounter{secnumdepth}{1}
 \setcounter{section}{1}
\section*{一、Monte Carlo integration}
\subsection{问题描述}
    数值计算积分
    \begin{equation}
        I(f)=\int_0^1 \sin(x)dx = \mathbb{E}(\sin X)
    \end{equation}
    并分析Monte Carlo半阶收敛性质。
\subsection{数值计算}
    在[0,1]中均匀取$N$个点,计算$\mathbb{E}(\sin X)=\frac{1}{N}\sum_{i=1}^N \sin X_i$,重复$M$次。
    计算平均误差$$\mathbb{E}(\vert e_n \vert)=\frac{1}{M}\sum_{j=1}^M\vert \bar{X}_j - (1-\cos(1)) \vert$$
    取$N=\{10,100,1000,10000,100000\}$,计算收敛阶
    $$o_k=\frac{\log(e_{n_{k+1}}/e_{n_k})}{\log(n_{k+1}/n_k)}$$
    结果如下表。
    \begin{table}[H]
        \caption{Monte Carlo积分误差即收敛阶}
        \centering
        \bigskip
        \begin{small}
        \begin{tabular}{|c|c|c|}
            \hline
            N & error &order   \\
            \hline
            10  &0.65872412 &NaN        \\
            100  &0.20495824  &0.50703819 \\
            1000  &0.06347936  &0.50903286  \\
            10000  &0.01962411  &0.50984243  \\
            100000  &0.00630419  &0.49316046  \\
            \hline
        \end{tabular}
    \end{small}
    \end{table}
 \setcounter{section}{2}
\section*{二、 limit process}
\subsection{问题描述}
    数值分析Binomial,Possion,Normal distribution 的极限性质。
\subsection{数值计算}
    数值计算Possion分布$P(\lambda)$的pmf,舍弃掉密度小于$1e-6$的点。
    计算Binomial分布$B(N,\lambda/N)$的pmf,并计算二者的偏差
    \begin{equation}
        loss_{\lambda,N}=\sum_{i=0}^\infty \vert \mathbb{P}(P(\lambda) = i) - \mathbb{P}(B(N,\lambda/N)=i)\vert
    \end{equation}
    通过增大$N$可以找到使得$loss_{\lambda,N}<\epsilon$的最小的$N$。
    取$\epsilon = 0.01$,结果如下。
    \begin{table}[H]
        \caption{Binomial分布逼近Possion分布要求的N}
        \centering
        \bigskip
        \begin{small}
        \begin{tabular}{|c|c|c|c|}
            \hline
            $\lambda$ & $N/\lambda$ &$\lambda$ & $N/\lambda$    \\
            \hline
            1  &56 &6  &48.2      \\
            2  &46  &7  & 49.1\\
            3  &50.6  &8 & 49.6 \\
            4  &50.7  &9 & 49.6 \\
            5  &49.6  &10 &49.5 \\
            \hline
        \end{tabular}
    \end{small}
    \end{table}
    可以看出当$p \simeq 0.02$时,大致有$loss<0.01$。\par
    对正态分布$N(0,1)$计算
    \begin{equation}
        p_{\lambda,i}=\int_{x_i-1/(2\sqrt{\lambda})}^{x_i+1/(2\sqrt{\lambda})}\frac{1}{\sqrt{2\pi}}e^{-x^2/2}dx
    \end{equation}
    其中$x_i = \frac{i-\lambda}{\sqrt{\lambda}}$.计算$loss_{\lambda}=\sum_{i=0}^\infty \vert \mathbb{P}(P(\lambda) = i) -p_{\lambda,i} \vert$。
    通过增大$\lambda$来减少$loss_{\lambda}$。取$\epsilon = 0.05$,在$\lambda = 26$时有$loss_{\lambda} <\epsilon$。
\subsection{画图}
    \begin{figure*}[ht]
        \centering
        \foreach \i in {1,2,3}{
            \subfigure[$\lambda$=\i]{\includegraphics[width=0.3\linewidth]{b2p_\i.png}}
        }
        \caption*{Binomial逼近Possion}
    \end{figure*}
    \begin{figure*}[ht]
        \centering
        \includegraphics[width=0.3\linewidth]{p2n.png}
        
        \caption*{Possion逼近正态分布}
    \end{figure*}
    
    \setcounter{section}{3}
\section*{三、 代码}
\subsection{数值积分}
\lstset{language=Python}
\begin{lstlisting}
import numpy as np

N=[10,100,1000,10000,100000]
nlist=np.array(N)
dlist=np.array([])
olist=np.array([])

for n in nlist:
    d=0
    for k in range(1000):
        x=np.random.rand(n)
        y=np.sin(x)
        i=sum(y)/n
        d+=abs(i-(1-np.cos(1)))
    dlist=np.append(dlist,d/100)

olist=np.log(dlist[0:-1]/dlist[1:])/np.log(nlist[1:]/nlist[0:-1])
print('order:',olist)
print('error',dlist)
\end{lstlisting}

\subsection{分布的极限}
\lstset{language=Matlab}
\begin{lstlisting}
    global epsilon;
    epsilon = 1e-6;
    ratiob=[];
    
    %% binomial逼近possion
    for lambda=1:10
        t=possion(lambda);
        nmin=t(1,1);
        nmax=t(end,1);
        loss=1;
        N=lambda+1;
    
        while(loss > 0.01)
            N=N+1;
            p=lambda/N;
            b=binomial(N,p,[nmin,nmax]);
            while(b(end,1)<nmax)
                b(end+1,:)=[b(end,1)+1;0];
            end
            loss=sum(abs(t(:,2)-b(:,2)));
        end
        ratiob(end+1)=N/lambda;
    end
    ratiob
    
    %% possion 逼近Guassian
    lambda = 10;
    gaussianpdf=@(mu,sigma,x)1/sqrt(2*pi*sigma)*exp(-(x-mu).^2/2/sigma^2);
    
    loss=1;
    while(loss > 0.05)
        lambda=lambda+1;
        t=possion(lambda);
        s=sqrt(lambda);
        x=(t(:,1)-lambda)/s;
        probg=zeros(size(x));
        parfor i=1:numel(x)
            probg(i)=integral(@(xx)gaussianpdf(0,1,xx),x(i)-1/s/2,x(i)+1/s/2);
        end
        loss=sum(abs(t(:,2)-probg));
    end
    
    lambda
    
    %% 画图
    figure(10);
    x=-3:0.01:3;
    y=gaussianpdf(0,1,x);
    plot(x,y);
    for lambda = 10:10:30
        t=possion(lambda);
        s=sqrt(lambda);
        x=(t(:,1)-lambda)/s;
        y=t(:,2)*s;
        hold on;
        plot(x,y);
    end
    legend('N(0,1)','possion(10)','possion(20)','possion(30)');
    
    for lambda=1:3
        figure(lambda);
        t=possion(lambda);
        nmin=t(1,1);
        nmax=t(end,1);
        plot(t(:,1),t(:,2));
        for N= (30:10:50) * lambda
            hold on;
            b=binomial(N,lambda/N,[nmin,nmax]);
            plot(b(:,1),b(:,2));
        end
        legend(['possion(',num2str(lambda),')'],['N=',num2str(30*lambda)],...
            ['N=',num2str(40*lambda)],['N=',num2str(50*lambda)])
    end
    
    function mf=possion(lambda)
        global epsilon
        i=0;
        mf=[];
        p=exp(-lambda);
        while(p<epsilon)
            i=i+1;
            p=p*lambda/i;
        end
        
        while(p>=epsilon)
            mf(end+1,:)=[i;p];
            i=i+1;
            p=p*lambda/i;
        end
    end
    
    function mf=binomial(n,p,range)
        q=1-p;
        nmin=max(0,range(1));
        nmax=min(n,range(2));
        mf=zeros(nmax-nmin+1,2);
        prob=nchoosek(n,nmin)*p^nmin*q^(n-nmin);
        for i = 1:nmax-nmin+1
            mf(i,:)=[i-1+nmin;prob];
            prob=prob*p/q/(i+nmin)*(n-i+1-nmin);
        end
    end    
\end{lstlisting}

\bibliographystyle{ieeetr}
\bibliography{refer}

\end{document}